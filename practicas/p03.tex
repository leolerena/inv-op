\documentclass[12pt]{article}
\usepackage[utf8]{inputenc}
\usepackage[spanish]{babel}
\usepackage[]{tcolorbox}
\usepackage{varwidth}   %% provides varwidth environment
\usepackage{empheq}
\usepackage{adjustbox}


\newcommand*{\cajados}[1]{\noindent\fbox{%
\parbox{\textwidth}{%
    #1
}%
}}
\newcommand{\caja}[1]{\begin{empheq}[box=\fbox]{alignat*=8}
    #1
\end{empheq}}

\newcommand*{\objetivodos}[2]{#1 x_{1} + #2 x_{2}}
\newcommand*{\objetivotres}[3]{#1 x_{1} + #2 x_{2} + #3x_{3}}
\newcommand*{\restdos}[3]{#1 x_{1} + #2 x_{2} \le #3}
\newcommand*{\irestdos}[3]{#1 x_{1}  #2 x_{2} = #3}
\newcommand*{\iresttres}[4]{#1 x_{1}  #2 x_{2} #3x_{3} = #4}

\newcommand*{\resttres}[4]{#1 x_{1} + #2 x_{2} + #3 x_{3} \le #4}
\newcommand*{\restuno}[2]{#1 x_{1} \le #2}
\newcommand{\posit}[1]{x_{#1} \ge 0}

\newcommand{\R}{\mathbb{R}}
\usepackage[]{mdframed}

%TCIDATA{TCIstyle=Article/art4.lat,jart,sw20jart}
\usepackage{amsmath,amsthm,amssymb, amsfonts, amscd, enumerate, multicol}
%TCIDATA{OutputFilter=LATEX.DLL}
%TCIDATA{LastRevised=Thu Feb 05 14:33:37 2004}
%TCIDATA{<META NAME="GraphicsSave" CONTENT="32">}
%TCIDATA{Language=Spanish}

%\input tcilatex
\setlength{\textheight}{25cm}
\setlength{\textwidth}{16.5cm}
\setlength{\topmargin}{-1.5cm}
\setlength{\oddsidemargin}{0cm}


\begin{document}
    \begin{center}
        {\bf UNIVERSIDAD DE SAN ANDR\'ES}
        \medskip
        
        {\bf Investigación operativa.}
        \medskip
        
        {\bf Ejercitaci\'on 3: Programación no lineal.}
        \medskip


        \begin{enumerate}
            \item Resolver los siguientes problemas de programación no-lineal mediante la función
            \emph{minimize()} en Python. 
            Para asegurarse de que llegan al óptimo global, aplicar la técnica
            de semillas randomizadas.
            \begin{enumerate}[a)]
                \item 
                    \caja{
                        \min  (x_{1} -2)^{2} + & (x_{1}-1)(x_{2} -1) + (x_{2} -3)^{2} \\
                        \\
                        & x_{2} = 2x_{1} + 2 \\
                        & x_{1}^{2} + x_{2}^{2} \le 8
                    }
                
                \item 
                    \caja{
                        \min \tfrac{1}{5}(x_{1} - 1)^{6} - \tfrac{5}{4}(x_{1} - 1)^{4} + 2(x_{1} -1)^{2} + \tfrac{1}{4}(x_{2} - 3)^{6} -\tfrac{5}{4}(x_{2} - 3)^{4} + 
                        2(x_{2}  - 3)^{2}
                    }
            \end{enumerate}
            \medskip 
            \medskip 
            \textbf{Recuerdo:}
            Dados dos puntos $(x_{0},y_{0}), (x_{1},y_{1}) \in \R^{2}$ su distancia se calcula como:
            \[
                d((x_{0},y_{0}), (x_{1},y_{1})) = \sqrt[2]{(x_{0} - x_{1})^{2} + (y_{0} - y_{1})^{2}}
            \]


            \item Una empresa quiere determinar dónde debe localizar su centro de distribución. La posición en el plano $x-y$ (en kilómetros) de sus cuatro clientes principales y el número de envíos anuales que deben hacer a cada cliente están dados en la siguiente tabla. 
            \begin{center}
                \begin{adjustbox}{width = 0.50 \textwidth}
                  \begin{tabular}{|c |c | c | c|}
                    \hline
                     & \multicolumn{2}{|c|}{ Coordenadas.} & \\
                     \hline 
                     Cliente & $x$ & $y$ & Cant. envíos. \\
                     \hline 
                     1 & 5 & 10 & 200 \\
                     2 & 10 & 5 & 150 \\
                     3 & 0 & 12 & 200 \\
                     4 & 12 & 0 & 300 \\
                    \hline 
                  \end{tabular}
                \end{adjustbox}
             \end{center}

            ¿Dónde debe instalar el centro de distribución si quiere minimizar la distancia total de envíos que debe hacer por año? 
            Formular un programa no-lineal para encontrar el punto óptimo. 
            
            \item Una compañía hace publicidades de sus productos en dos tipos de programas televisivos:
            telenovelas y partidos de fútbol. Cada publicidad en una telenovela cuesta \$50, y cada publicidad en un partido de fútbol cuesta \$100. 

            Si se hacen $S$ publicidades en telenovelas lo van a ver $0,5 \  \sqrt[]{S}$ millones de hombres y $2\sqrt{S}$ millones de mujeres. 
            Si se hacen $F$ publicidades en partidos de fútbol, serán vistas por $1. 7 \sqrt{F}$ millones de hombres, y $0.7 \sqrt{F}$ millones de mujeres. 
            La compañía quiere que por lo menos 40 millones de hombres y por lo menos 60
            millones de mujeres vean la propaganda. 

            Formular un programa no-lineal para cumplir con
            los requisitos de audiencia con costo mínimo. 

            \item Un joven ingeniero de una compañía ha sintetizado un nuevo fertilizante hecho a partir de dos materias primas. 
            Al combinar cantidades de las materias primas básicas $x_{1} y x_{2}$,
            la cantidad de fertilizante que se obtiene viene dada por
            \[
                Q(x_{1},x_{2}) = 4x_{1} + 2x_{2} - 0.5x_{1}^{2}- 0.25x_{2}^{2}
            \]  
            Se requieren \$480000 pesos por unidad de materia prima 1 y \$300000 pesos por cada unidad de materia prima 2 que se empleen en la fabricación del fertilizante (en estas cantidades se incluyen los costos de las materias primas y los costos de producción). 
            Si la compañía dispone de \$24.000.000 pesos para la producción de materias primas, plantear el problema para determinar la cantidad de materia prima de forma que se maximice la cantidad de fertilizante.

            \item Una empresa produce frigoríficos y ha firmado un contrato para suministrar al menos 150 unidades en tres meses, 50 unidades al final del primer mes, 50 al final del segundo y 50 al final del tercero. 
            El coste de producir una cantidad de frigoríficos en cualquier mes es su
            cuadrado. 
            La empresa puede producir si lo desea más frigoríficos de los que necesita en
            cualquier mes y guardarlos para el siguiente, siendo el coste de almacenaje de \$12000 pesos por unidad al mes. 
            Suponiendo que no hay inventario inicial, formular el programa adecuado para determinar el número de frigoríficos que deben producirse cada mes, para
            minimizar el costo total.

            % \item Se poseen los siguientes datos sobre una población animal:  
            % en un año $t_{i}$ se observó una cantidad $y_{i}$ de población de este animal, esto se hizo a lo largo de cinco años. 
            % Queremos ajustar a un modelo $y = ae^{bt}$ para predecir la población $y$ en base al tiempo $t$.
            % \begin{center}
            %     \begin{adjustbox}{width = 0.30 \textwidth}
            %       \begin{tabular}{|c |c | c | c|c|c|}
            %         \hline
            %          $t_{i}$ & 1 & 2 & 4 & 5 & 8 \\
            %          \hline
            %          $y_{i}$ & 4 & 4 & 6 & 11 & 22 \\
            %         \hline 
            %       \end{tabular}
            %     \end{adjustbox}
            %  \end{center}
            % Determinar el error mínimo incurrido.

            % \emph{Sugerencia.} Como alternativa pueden usar la medición del error cuadrático medio.

            \item Una compañía planea gastar \$10.000.000 de pesos en publicidad. 
            Se sabe que un minuto de publicidad en televisión cuesta \$3.000.000 de pesos y \$1.000.000 de pesos en la radio. 
            Si la empresa compra $x$ minutos de publicidad en televisión e $y$ minutos en la radio, su ingreso, en pesos, está dado por
            \[
             I(x,y) = -2x^{2} - y^{2} + xy + 8x + 3y
            \]
            ¿Cómo puede la empresa maximizar sus ingresos?

            \item Una compañía petrolera debe determinar cuántos barriles de petróleo hay que extraer en los próximos dos años. 
            Si la compañía extrae $x_{1}$ millones de barriles durante un año, se
            pondrá vender cada barril a $30000-x_{1}$ pesos. Si extrae $x_{2}$ millones de barriles durante el segundo año, se podrá vender cada barril a $35000-x_{2}$ pesos. El costo para extraer $x_{1}$ millones de barriles en el primer año es de $x_{1}^{2}$
            millones de pesos y el costo para extraer $x_{2}$ millones de barriles durante el segundo año es de $2x_{2}^{2}$ millones de pesos. 
            Se puede obtener como máximo un total de 20 millones de barriles de petróleo, y se puede gastar como máximo 250 millones de pesos en la extracción. 

            Formular un problema de programación no lineal para ayudar a la empresa a maximizar
            sus ganancias para los próximos dos años.

            \item La funcion Sigmoidea es comunmente utilizada como función de activación
            en las redes neuronales. 
            Se define como:
            \[
             \sigma(z) = \frac{1}{1+e^{-z}}
            \]
            El modelo de regresión logística es un modelo de clasificación en el cual predecimos el valor de $y \in \{0,1\}$ a partir de $x$:
            \[
             \hat{y} = \sigma(wx + b)
            \]
            A partir de un conjunto de datos $X = \{x_{1},x_{2},\dots,x_{n}\}$ y sus etiquetas $\{y_{1},y_{2}, \dots, y_{n}\}$.
            Queremos optimizar $w$ y $b$ con tal de minimizar la siguiente función de pérdida.
            \[
             {\cal L} = -\frac{1}{N} \sum_{i=1}^{N} y_{i}\log(\hat{y_{i}}) + (1-y_{i}\log(1-\hat{y_{i}}))
            \]
            Esta función se conoce como entropía cruzada binaria.
            \begin{enumerate}[a)]
                \item Obtener la fórmula del gradiente de ${\cal L}(w,b)$.
                    \emph{Sugerencia.} Utilizar la regla de la cadena.
                \item Para el siguiente conjunto de datos realizar una iteración de descenso de gradiente.
                Utilizar un learning rate $\gamma = 0,05$ y $w=0, b =0$ como punto inicial.


                \begin{center}
                    \begin{adjustbox}{width = 0.15 \textwidth}
                      \begin{tabular}{|c|c|}
                         \hline
                        $x$ & $y$ \\
                        \hline 
                        0.52 &  1 \\
                        -1.27 & 0 \\
                        0.24 &  1 \\
                        -0.80 & 1 \\
                        0.02 &  1 \\
                        \hline 
                      \end{tabular}
                    \end{adjustbox}
                 \end{center}
            \end{enumerate}

            \item Optimizar por descenso de gradiente la siguiente función:
                 \[
                    f(x,y) = x^{4} + y^{4} - 4xy + 3
                 \]
                \begin{enumerate}[a)]
                    \item Realizar 10 iteraciones utilizando learning rates de $\{0.1, \  0.01, \  0,001\}$. 
                    Decidir cuál es mejor.

                    \item Utilizar distintos puntos iniciales como semilla y explorar como cam-
                    bian los mínimos alcanzados. 
                    Extraer conclusiones al respecto.
                \end{enumerate} 

            \item  Sean $x,y,z$ el largo, ancho y alto de una caja de cartón respectivamente.
                El volúmen de la caja se obtiene por medio de 
                \[
                    V(x,y,z) = xyz
                \]    
                Se cuentan $100 cm^{2}$ de cartón para armar una caja.
                \begin{enumerate}[a)]
                    \item Proponer un modelo de optimización no lineal para encontrar las
                    dimensiones de la caja de forma tal que el volúmen sea máximo.
                \end{enumerate}
        \end{enumerate}
    \end{center}
\end{document}