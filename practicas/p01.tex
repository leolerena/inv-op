\documentclass[12pt]{article}
\usepackage[utf8]{inputenc}
\usepackage[spanish]{babel}
\usepackage[]{tcolorbox}
\usepackage{varwidth}   %% provides varwidth environment
\usepackage{empheq}
\usepackage{adjustbox}


\newcommand*{\cajados}[1]{\noindent\fbox{%
\parbox{\textwidth}{%
    #1
}%
}}
\newcommand{\caja}[1]{\begin{empheq}[box=\fbox]{alignat*=8}
    #1
\end{empheq}}

\newcommand*{\objetivodos}[2]{#1 x_{1} + #2 x_{2}}
\newcommand*{\objetivotres}[3]{#1 x_{1} + #2 x_{2} + #3x_{3}}
\newcommand*{\restdos}[3]{#1 x_{1} + #2 x_{2} \le #3}
\newcommand*{\irestdos}[3]{#1 x_{1}  #2 x_{2} = #3}
\newcommand*{\iresttres}[4]{#1 x_{1}  #2 x_{2} #3x_{3} = #4}

\newcommand*{\resttres}[4]{#1 x_{1} + #2 x_{2} + #3 x_{3} \le #4}
\newcommand*{\restuno}[2]{#1 x_{1} \le #2}
\newcommand{\posit}[1]{x_{#1} \ge 0}

\newcommand{\R}{\mathbb{R}}
\usepackage[]{mdframed}

%TCIDATA{TCIstyle=Article/art4.lat,jart,sw20jart}
\usepackage{amsmath,amsthm,amssymb, amsfonts, amscd, enumerate, multicol}
%TCIDATA{OutputFilter=LATEX.DLL}
%TCIDATA{LastRevised=Thu Feb 05 14:33:37 2004}
%TCIDATA{<META NAME="GraphicsSave" CONTENT="32">}
%TCIDATA{Language=Spanish}

%\input tcilatex
\setlength{\textheight}{25cm}
\setlength{\textwidth}{16.5cm}
\setlength{\topmargin}{-1.5cm}
\setlength{\oddsidemargin}{0cm}


\begin{document}
    \begin{center}
        {\bf UNIVERSIDAD DE SAN ANDR\'ES}
        \medskip
        
        {\bf Investigación operativa.}
        \medskip
        
        {\bf Ejercitaci\'on 1: Introducción a Python y optimización.}
        \medskip


        \begin{enumerate}
            \item Escribir un código que imprima a la consola las siguientes frases u operaciones \\
            matemáticas:
            \begin{enumerate}[a)]
                \item "Hola mundo".
                \item 2+3
                \item 2*3
                \item $2^{3}$
                \item $\dfrac{2}{3}$
                \item $2 \% 3$ (el resto de dividir 2 por 3).
            \end{enumerate}

            \item Escribir un código que imprima todos los números pares entre 0 y 31, utilizando \textbf{for loops}.

            \item Escribir un código que compute el promedio de la lista de números [1,32,53,14,55,36,27].
            Hacerlo de dos maneras distintas:
            \begin{enumerate}[a)]
                \item Mediante \textbf{for loops.}
                \item Usando la función \textbf{np.mean}.
            \end{enumerate}

            \item Escribir una función que tome como input dos números $x_{1},x_{2}$ e imprima a la consola la
            suma de esos dos números.

            \item 
            (Introducción a numpy.) El siguiente ejercicio es para que nos familiaricemos con la sintaxis y las funciones de numpy.
            \begin{enumerate}[a)]
                \item Importar la librería numpy con el comando “import numpy as np”.
                \item Considerar la matriz
                $A = [[1, 2, 0],
                [3, 0, 4],
                [1, 0, 3]]$.
                \item Transformar en un array de numpy con el comando “A = np.array(A)”.
                \item Corroborar que los comandos $A[0][0]$ y $A[0, 0]$ devuelven el primer elemento en la
                primera fila, y primera columna.
                \item Corroborar que el comando A[:,1] devuelve la segunda columna.
                \item Corroborar que los comandos A[1] y A[1,:] devuelven la segunda fila.
                \item Corroborar que el comando A[:, -1] devuelve la última columna.
                \item Corroborar que A[0:2] y A[:,0:2] devuelven las primeras dos filas y las primeras dos
                columnas respectivamente.
                \item ¿Qué devuelven los comandos A[-1, -1], A[0:2], A[0:2, 0], A[0:2, 0:2]?
            \end{enumerate}
            
            \item (Producto interno entre vectores).
                El siguiente ejercicio nos propone implementar el producto interno de vectores.
                \begin{enumerate}[a)]
                    \item Escribir un for loop que dadas las dos listas:
                    \[
                        A = [2, 10, 16, 2, 4, 12, 24, 100],  \ 
                        B = [5, 2, 5, 2, 1, 2, 1, 0.5]
                    \]
                    
                    sume la multiplicación coordenada a coordenada de todos sus elementos, es decir:
                    \[
                        A[0]*B[0] + A[1]*B[1] + A[2]*B[2] + ... = 2*5 + 10*2 + 16*5 ...
                    \]
                    Esta operación es el \emph{producto interno} entre dos vectores (que en python representamos por listas.)

                    \item Importar la librería numpy y transformar ambas listas en arrays de numpy con las
                    siguientes líneas de código:

                     Import numpy as np

                     A = np.array(A)

                     B = np.array(B)
                    
                    \item Corroborar que ahora el comando “np.dot(A, B)” da el mismo resultado que en el ítem a).
                \end{enumerate}

                \item El siguiente ejercicio nos propone implementar el producto de matrices.
                \begin{enumerate}[a)]
                    \item Considerar las matrices
                    \[
                        A = [[1, 2, 0],
                        [3, 0, 4],
                        [1, 0, 3]], \ \ 
                        B = [[3, 1, 1],
                        [1, 2, 0],
                        [2, 4, 1]]
                    \]
                    y transformarlas en arrays de numpy con los siguientes comandos:

                    A = np.array(A)

                    B = np.array(B)

                    \item Hacer un doble for loop y utilizar el ejercicio anterior para construir la matriz A*B.

                    \item Multiplicar las matrices usando el comando abreviado de numpy $A @ B$.
                \end{enumerate}

                \item  Escribir una función que tome como entrada dos números $x_{1},x_{2}$ y determine si están en el conjunto factible definido por las siguientes restricciones lineales.
                La salida debe ser un booleano.
                \caja{
                    \restdos{2}{3}{24} \\ 
                    \posit{1} \\
                    \posit{2}
                }
                Usando esta función determinar si los siguientes puntos están en el conjunto factible o no.
                \begin{enumerate}[a)]
                    \item $(x_{1},x_{2}) = (1,1)$
                    \item $(x_{1},x_{2}) = (12,8)$
                    \item $(x_{1},x_{2}) = (4,8)$
                    \item $(x_{1},x_{2}) = (-1,0)$
                    \item $(x_{1},x_{2}) = (-2,2)$
                \end{enumerate}
                    
                \item Para los siguientes problemas de optimización encontrar gráficamente el valor óptimo y los puntos donde alcanza este valor.
                Para hacer esto pueden graficar las funciones objetivo en el
                rango de interés y visualmente
                identificar el punto y valor óptimo. 
                Para los casos donde haya más de un óptimo, reportar todos los puntos óptimos que existan. Para los casos en que no existe el óptimo, explicar por qué sucede esto.
                \begin{enumerate}[a)]
                    \item 
                        \caja{\min \ x^{2} \\
                        \\
                        x \in \R.
                        }
                    \item  
                        \caja{\min \  x^{2} \\
                        \\
                        x \ge 2}
                    \item 
                        \caja{\max \ x \\
                            \\
                              0 \le x}
                    \item 
                        \caja{\max x \\
                                \\
                              x \le 0 \\
                              2 \le x}
                    \item 
                        \caja{\min \ \cos(x) \\ 
                        \\
                        0 \le x \\
                        x \le 1}
                    \item 
                        \caja{\min \  \cos(x) \\
                        \\
                        0 \le x \\
                        x \le 10}
                \end{enumerate}
            \item Para los siguientes problemas de optimización, identificar: si el problema es de
            maximización o minimización, las variables de decisión, la función objetivo, las
            restricciones de igualdad, si las hay, y las restricciones de desigualdad, si las hay. 
            Hallar, si es posible, los valores que maximicen/minimicen las funciones objetivos según corresponda
            utilizando la librería \emph{PICOS}.
            \begin{enumerate}[a)]
                \item 
                    \caja{
                        \max \ \objetivodos{3}{4} \\
                        \\
                        \restdos{}{}{10} \\ 
                        \restdos{}{0.7}{11} \\
                        \posit{1} \\
                        \posit{2}
                    }
                \item 
                    \caja{
                        \max \ \objetivodos{3}{4} \\
                        \\
                        \restdos{}{}{10} \\ 
                        \restdos{}{0.7}{11} \\
                        \irestdos{}{-}{0} \\ 
                        \posit{1} \\
                        \posit{2}
                    }
                    \item 
                    \caja{
                        \max \ \objetivodos{30}{100} \\
                        \\
                        \restdos{}{}{7} \\ 
                        \restdos{4}{10}{40} \\
                        \restuno{-10}{-30} \\
                        \posit{1} \\
                        \posit{2}
                    }

                    \item 
                    \caja{
                        \min \ \objetivotres{15}{10}{20} \\
                        \\
                        \resttres{0.1}{0.1}{0.7}{60} \\
                        \iresttres{0.1}{+0.2}{+0.4}{30} \\
                        \iresttres{0.45}{+0.25}{+0.3}{40} \\
                        \posit{1} \\
                        \posit{2} \\
                        \posit{3} \\
                    }
            \end{enumerate}
        \end{enumerate}


    \end{center}
\end{document}