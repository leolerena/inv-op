\documentclass{beamer}
\usefonttheme{professionalfonts}% Now beamer didn't modify the math fonts
\usepackage{geometry}
\usepackage[]{tcolorbox}
\usepackage{varwidth}   %% provides varwidth environment
\usepackage{empheq}
\usepackage{adjustbox}


\newcommand*{\cajados}[1]{\noindent\fbox{%
\parbox{\textwidth}{%
    #1
}%
}}
\newcommand{\caja}[1]{\begin{empheq}[box=\fbox]{alignat*=8}
    #1
\end{empheq}}

\newcommand*{\objetivodos}[2]{#1 x_{1} + #2 x_{2}}
\newcommand*{\objetivotres}[3]{#1 x_{1} + #2 x_{2} + #3x_{3}}
\newcommand*{\restdos}[3]{#1 x_{1} + #2 x_{2} \le #3}
\newcommand*{\irestdos}[3]{#1 x_{1}  #2 x_{2} = #3}
\newcommand*{\iresttres}[4]{#1 x_{1}  #2 x_{2} #3x_{3} = #4}

\newcommand*{\resttres}[4]{#1 x_{1} + #2 x_{2} + #3 x_{3} \le #4}
\newcommand*{\restuno}[2]{#1 x_{1} \le #2}
\newcommand{\posit}[1]{x_{#1} \ge 0}

\newcommand{\R}{\mathbb{R}}
% \usepackage{tikz-cd}
\beamertemplatenavigationsymbolsempty
\setbeamertemplate{footline}[page number]{}
\setbeamerfont{footline}{size=\footnotesize}
\usepackage{tcolorbox}
\usepackage{biblatex}
% \usepackage{tikzcd}
\usetheme{Copenhagen}
\usepackage{tikz}

\definecolor{asparagus}{rgb}{0.53, 0.66, 0.42}


\title[]{Clase tutorial 08: Repaso.}
% \subtitle{Workshop de LOREL 2024.}
\date{25 de Septiembre de 2024.}
\author[]{ Investigación operativa.}
\institute{Universidad de San Andrés}
% \titlegraphic{\hfill\includegraphics[height=1.5cm]{logo.pdf}}

\begin{document}
\maketitle

\begin{frame}{Objetivos de la clase de hoy.}
  La idea de la clase de hoy es ver los siguientes temas:
  \begin{itemize}
    \item Repasar temas de Programación lineal y no lineal para el parcial.
  \end{itemize}
\end{frame}


\begin{frame}[fragile]{Problema 1.}
  Una pareja, Micaela y Gastón, quiere dividir las principales tareas del hogar (ir de compras,
  cocinar, lavar platos y lavar ropa) entre los dos, de manera que cada uno tenga dos
  obligaciones y el tiempo total para hacer estas tareas sea mínimo. La eficiencia en cada
  una de las tareas difiere entre ellos; la siguiente tabla proporciona el tiempo que cada uno
  necesita para cada tarea. Formule el problema como un programa entero y encuentre el
  equipo óptimo.

  \begin{center}
    \begin{adjustbox}{width = 0.85 \textwidth}
        \begin{tabular}{|c |c |c|c|c|}
          \hline
          & \multicolumn{4}{|c|}{Horas necesarias por semana.} \\ 
          \hline
          & Compras & Cocinar & Lavar platos & Lavar ropa \\
          \hline
          Micaela & 4,5 horas & 7,8 horas & 3,6 horas & 2,9 horas \\ 
          Gastón & 4,9 horas & 7,2 horas & 4,3 horas & 3,1 horas \\
          \hline 
        \end{tabular}
    \end{adjustbox}
  \end{center}

  
\end{frame}

\begin{frame}[fragile]{Problema 2}
  Una ciudad debe construir un barrio nueva por completo que cubrirá varios kilometros cuadrados.
  Una de las decisiones a tomar es la ubicación de las estaciones de bomberos que le asignaron al barrio.
  Para propósitos de planificación la ciudad dividió en cinco secciones tal que cada sección puede tener como máximo una estación de bomberos.
  Cada estación debe responder los llamados de todos los otros sectores no solo del cual se encuentra localizado.
  Entonces las decisiones a tomar es: ¿qué sectores albergarán una estación de bomberos?
  El objetivo es minimizar el promedio global de los tiempos de respuesta a los incendios.
\end{frame}

\begin{frame}[fragile]{Problema 2.}
  \begin{center}
    \begin{adjustbox}{width = 0.95 \textwidth}
        \begin{tabular}{|c |c |c|c|c|c|}
          \hline
          & \multicolumn{5}{|c|}{Tiempo de respuesta en minutos.} \\ 
          \hline
          Ubicación de la estación & 1 & 2 & 3 & 4 & 5 \\
          \hline
          1 & 5 & 12 & 30 & 20 & 15 \\ 
          2 & 20 & 4 & 15 & 10 & 25 \\
          3 & 15 & 20 & 6 & 15 & 12 \\
          4 & 25 & 15 & 25 & 4 & 10 \\
          5 & 10 & 25 & 15 & 12 & 5 \\
          \hline
          Frecuencia de emergencias por día & 2  & 1  & 3 & 1 & 3 \\
          \hline 
        \end{tabular}
    \end{adjustbox}
  \end{center}
\end{frame}

\begin{frame}[fragile]{Problema 3.}
En una fábrica se producen dos modelos de un producto. 
Si $x$ e $y$ son las cantidades de
cada modelo que se arman por día, la función de utilidad es $U (x, y) = 3000x^{0,5}y^{0,6}$. 
Se
dispone diariamente de \$ 110000 para producir y el costo de cada producto del primer
modelo es \$ 1000 y del segundo \$ 1500. 
Calcular los valores de x e y que maximizan la
utilidad e indicar cuál es esa utilidad máxima.
\end{frame}

\end{document}
