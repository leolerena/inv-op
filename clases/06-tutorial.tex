\documentclass{beamer}
\usefonttheme{professionalfonts}% Now beamer didn't modify the math fonts
\usepackage{geometry}
\usepackage[]{tcolorbox}
\usepackage{varwidth}   %% provides varwidth environment
\usepackage{empheq}
\usepackage{adjustbox}


\newcommand*{\cajados}[1]{\noindent\fbox{%
\parbox{\textwidth}{%
    #1
}%
}}
\newcommand{\caja}[1]{\begin{empheq}[box=\fbox]{alignat*=8}
    #1
\end{empheq}}

\newcommand*{\objetivodos}[2]{#1 x_{1} + #2 x_{2}}
\newcommand*{\objetivotres}[3]{#1 x_{1} + #2 x_{2} + #3x_{3}}
\newcommand*{\restdos}[3]{#1 x_{1} + #2 x_{2} \le #3}
\newcommand*{\irestdos}[3]{#1 x_{1}  #2 x_{2} = #3}
\newcommand*{\iresttres}[4]{#1 x_{1}  #2 x_{2} #3x_{3} = #4}

\newcommand*{\resttres}[4]{#1 x_{1} + #2 x_{2} + #3 x_{3} \le #4}
\newcommand*{\restuno}[2]{#1 x_{1} \le #2}
\newcommand{\posit}[1]{x_{#1} \ge 0}

\newcommand{\R}{\mathbb{R}}
% \usepackage{tikz-cd}
\beamertemplatenavigationsymbolsempty
\setbeamertemplate{footline}[page number]{}
\setbeamerfont{footline}{size=\footnotesize}
\usepackage{tcolorbox}
\usepackage{biblatex}
% \usepackage{tikzcd}
\usetheme{Copenhagen}
\usepackage{tikz}

\definecolor{asparagus}{rgb}{0.53, 0.66, 0.42}


\title[]{Clase tutorial 06: Programación entera.}
% \subtitle{Workshop de LOREL 2024.}
\date{10 de Septiembre de 2024.}
\author[]{ Investigación operativa.
  }
\institute{Universidad de San Andrés}
% \titlegraphic{\hfill\includegraphics[height=1.5cm]{logo.pdf}}

\begin{document}
\maketitle

\begin{frame}{Objetivos de la clase de hoy.}
  La idea de la clase de hoy es ver los siguientes temas:
  \begin{itemize}
    \item Problemas no estructurados de programación no lineal y como resolverlos con Python.
    \item Optimización utilizando semillas aleatorias.
    \item Descenso del gradiente.
  \end{itemize}
\end{frame}

\begin{frame}[fragile]{Repaso.}
  Debemos tener en cuenta algunas diferencias entre la programación lineal y la programación no lineal.
  \pause 
  \begin{itemize}
    \item Las variables de decisión son reales pero al menos la función objetivo o alguna de las restricciones no es una función lineal.
    \pause 
    \item Los problemas pueden \textbf{no} tener restricciones.
    \pause 
    \item En la implementación usamos Scipy (dejamos de usar Picos por ahora).
    \pause 
    \item Para la implementación nos sirve recordar que vale lo siguiente:
      \[
        \min_{x \in X} f(x) = - \max_{x \in X} -f(x)
      \]
  \end{itemize}
\end{frame}

\begin{frame}[fragile]{Problema distancia aeropuerto.}
  Tres ciudades se ubican en los vértices de un triángulo equilátero.

  Un aeropuerto debe construirse en un punto que minimice la distancia a las tres ciudades.

  Construir un problema de programación no lineal para tomar la decisión.
\end{frame}

\begin{frame}[fragile]{Problema precio de barra de caramelos.}
  Se considera cargar un precio de venta de una barra de caramelo en el rango de los siguientes precios: como mínimo \$1.10 y como máximo \$ 1.50.
  
  Para precios de \$1.10, \$1.30 y \$1.50 el departamento de marketing estimó la demanda en tres regiones donde se va a vender (ver tabla adjunta).

  \begin{center}
    \begin{adjustbox}{width = 0.9 \textwidth}
        \begin{tabular}{|c |c | c | c|}
          \hline
          & \multicolumn{3}{|c|}{Demanda (en miles)} \\ 
          \hline 
          Precio & Región 1 & Región 2 &  Región 3 \\ 
          \hline
          Bajo (1.10)  & 35 & 32 & 24 \\
          Medio (1.30) & 32 & 27 & 17 \\
          Alto (1.50)  & 22 & 16 & 9 \\
          \hline 
        \end{tabular}
    \end{adjustbox}
  \end{center}

  Teniendo en cuenta estos datos. 
  ¿Qué precio maximiza la ganancia?
\end{frame}

\begin{frame}[fragile]{Problema de semillas aleatorias.}
  Minimizar la siguiente función en Python: 
  \[
    f(x,y) = 
    0.22x^6 - 1.25x^4 + 2  x^2 + 0.22y^6 - 1.25y^4 + 2y^2
  \]
  usar varias semillas aleatorias diferentes y comparar los resultados de optimizar la función con estas semillas.
\end{frame}

\begin{frame}[fragile]{Problema de descenso del gradiente.}
  Optimizar usando la técnica de descenso del gradiente la siguiente función: 
  \[
    f(x,y) = 6x^{2} + 3y^{4} - 4xy + 2
  \]
  
  Realizar 1000 iteraciones variando el valor de $\gamma$ dentro de $\{0.001,0.00001,0.000001\}$.
\end{frame}

\end{document}
