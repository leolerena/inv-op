\documentclass{beamer}
\usefonttheme{professionalfonts}% Now beamer didn't modify the math fonts
\usepackage{geometry}
\usepackage[]{tcolorbox}
\usepackage{varwidth}   %% provides varwidth environment
\usepackage{empheq}
\usepackage{adjustbox}


\newcommand*{\cajados}[1]{\noindent\fbox{%
\parbox{\textwidth}{%
    #1
}%
}}
\newcommand{\caja}[1]{\begin{empheq}[box=\fbox]{alignat*=8}
    #1
\end{empheq}}

\newcommand*{\objetivodos}[2]{#1 x_{1} + #2 x_{2}}
\newcommand*{\objetivotres}[3]{#1 x_{1} + #2 x_{2} + #3x_{3}}
\newcommand*{\restdos}[3]{#1 x_{1} + #2 x_{2} \le #3}
\newcommand*{\irestdos}[3]{#1 x_{1}  #2 x_{2} = #3}
\newcommand*{\iresttres}[4]{#1 x_{1}  #2 x_{2} #3x_{3} = #4}

\newcommand*{\resttres}[4]{#1 x_{1} + #2 x_{2} + #3 x_{3} \le #4}
\newcommand*{\restuno}[2]{#1 x_{1} \le #2}
\newcommand{\posit}[1]{x_{#1} \ge 0}

\newcommand{\R}{\mathbb{R}}
% \usepackage{tikz-cd}
\beamertemplatenavigationsymbolsempty
\setbeamertemplate{footline}[page number]{}
\setbeamerfont{footline}{size=\footnotesize}
\usepackage{tcolorbox}
\usepackage{biblatex}
% \usepackage{tikzcd}
\usetheme{Copenhagen}
\usepackage{tikz}

\definecolor{asparagus}{rgb}{0.53, 0.66, 0.42}


\title[]{Clase tutorial 06: Programación entera.}
% \subtitle{Workshop de LOREL 2024.}
\date{18 de Septiembre de 2024.}
\author[]{ Investigación operativa.}
\institute{Universidad de San Andrés}
% \titlegraphic{\hfill\includegraphics[height=1.5cm]{logo.pdf}}

\begin{document}
\maketitle

\begin{frame}{Objetivos de la clase de hoy.}
  La idea de la clase de hoy es ver los siguientes temas:
  \begin{itemize}
    \item Problemas no estructurados de programación no lineal y como resolverlos con Python.
    \item Descenso del gradiente.
  \end{itemize}
\end{frame}


\begin{frame}[fragile]{Problema de producción.}
  Una empresa produce caucho utilizado para neumáticos combinando tres ingredientes: \emph{caucho, aceite y carbono}. 
  El costo en centavos por kilo de cada ingrediente se proporciona en la siguiente tabla.
  \pause 

  \begin{center}
    \begin{adjustbox}{width = 0.35 \textwidth}
        \begin{tabular}{|c |c |}
          \hline
          Producto & Costo \\ 
          \hline
          Caucho   & 4  \\
          Aceite   & 1  \\
          Carbono  & 7  \\
          \hline 
        \end{tabular}
    \end{adjustbox}
  \end{center}

  
\end{frame}

\begin{frame}[fragile]{Problema de producción.}
  El caucho utilizado en los neumáticos de automóvil debe tener una \textbf{dureza} entre 25 y 35, una \textbf{elasticidad} de al menos 16 y una \textbf{resistencia a la tracción} de al menos 12. 
  \pause 
  Para fabricar un juego de cuatro neumáticos de automóvil, se necesitan 100 kilos de producto. 

  Para hacer un juego de cuatro neumáticos éste debe contener entre 25 y 60 kilos de caucho, al menos 5 kilos de aceite y al menos 50 kilos de carbono.

\end{frame}

\begin{frame}[fragile]{Problema de producción.}
  Si denotamos por: 
  \begin{itemize}
    \item $R$ la cantidad en kilos de caucho necesaria par construir cuatro neumáticos.
    \item $O$ la cantidad en kilos de aceite necesaria par construir cuatro neumáticos.
    \item $C$ la cantidad en kilos de carbono necesaria par construir cuatro neumáticos.
  \end{itemize}
  \pause 

 El análisis estadístico ha demostrado que la dureza, la elasticidad y la resistencia a la tracción de una mezcla de 100 kilos de caucho, aceite y carbono son las siguientes 
  \begin{align*}
    &\text{Resistencia tracción} =   12.5 + 0.1O + 0.001O^{2} \\ 
    &\text{Elasticidad} =  17 + 0.35R + 0.04O + 0.002R \\ 
    &\text{Dureza} =   34 + 0.1R + 0.06O + 0.3C + 0.001RO + 0.005O^{2}  + 0.001C^{2} \\ 
  \end{align*}
\end{frame}

\begin{frame}[fragile]{Problema de producción.}
  Plantear un problema de programación no lineal que minimize el costo y resolverlo utilizando Python.
\end{frame}

\begin{frame}[fragile]{Problema de descenso del gradiente.}
  Optimizar usando la técnica de descenso del gradiente la siguiente función: 
  \[
    f(x,y) = 6x^{2} + 3y^{4} - 4xy + 2
  \]
  
  Realizar 1000 iteraciones variando el valor de $\gamma$ dentro de $\{0.001,0.00001,0.000001\}$.
\end{frame}

\end{document}
