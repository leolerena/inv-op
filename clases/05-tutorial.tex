\documentclass{beamer}
\usefonttheme{professionalfonts}% Now beamer didn't modify the math fonts
\usepackage{geometry}
\usepackage[]{tcolorbox}
\usepackage{varwidth}   %% provides varwidth environment
\usepackage{empheq}
\usepackage{adjustbox}


\newcommand*{\cajados}[1]{\noindent\fbox{%
\parbox{\textwidth}{%
    #1
}%
}}
\newcommand{\caja}[1]{\begin{empheq}[box=\fbox]{alignat*=8}
    #1
\end{empheq}}

\newcommand*{\objetivodos}[2]{#1 x_{1} + #2 x_{2}}
\newcommand*{\objetivotres}[3]{#1 x_{1} + #2 x_{2} + #3x_{3}}
\newcommand*{\restdos}[3]{#1 x_{1} + #2 x_{2} \le #3}
\newcommand*{\irestdos}[3]{#1 x_{1}  #2 x_{2} = #3}
\newcommand*{\iresttres}[4]{#1 x_{1}  #2 x_{2} #3x_{3} = #4}

\newcommand*{\resttres}[4]{#1 x_{1} + #2 x_{2} + #3 x_{3} \le #4}
\newcommand*{\restuno}[2]{#1 x_{1} \le #2}
\newcommand{\posit}[1]{x_{#1} \ge 0}

\newcommand{\R}{\mathbb{R}}
% \usepackage{tikz-cd}
\beamertemplatenavigationsymbolsempty
\setbeamertemplate{footline}[page number]{}
\setbeamerfont{footline}{size=\footnotesize}
\usepackage{tcolorbox}
\usepackage{biblatex}
% \usepackage{tikzcd}
\usetheme{Copenhagen}
\usepackage{tikz}

\definecolor{asparagus}{rgb}{0.53, 0.66, 0.42}


\title[]{Clase tutorial 04: Programación entera.}
% \subtitle{Workshop de LOREL 2024.}
\date{4 de Septiembre de 2024.}
\author[]{ Investigación operativa.
  }
\institute{Universidad de San Andrés}
% \titlegraphic{\hfill\includegraphics[height=1.5cm]{logo.pdf}}

\begin{document}
\maketitle

\begin{frame}{Objetivos de la clase de hoy.}
  La idea de la clase de hoy es ver los siguientes temas:
  \begin{itemize}
    \item Traveling salesman.
    \item Repaso programación lineal.
  \end{itemize}
\end{frame}

\begin{frame}[fragile]{Problema de traveling salesman.}
  Sea la siguiente tabla con las distancias que hay entre 8 ciudades distintas (donde un guión - indica que no hay una arista entre ambas ciudades).

  \begin{center}
    \begin{adjustbox}{width = 0.9 \textwidth}
        \begin{tabular}{|c |c | c | c| c| c| c | c |}
          \hline
          Ciudad & 2 & 3 & 4 & 5 & 6 & 7 & 8 \\
          \hline 
          1 & 14 & 15 & — & — & — & — & 17 \\ 
          2 &  & 13 & 14 & 20 & — & — & 21 \\
          3 & & & 11  & 21 &  17 & 9  & 9 \\
          4 & & & & 11  & 10 & 8 & 20 \\
          5 & & & & &  15 &  18  & — \\
          6 & & & & & & 9 & — \\
          7 & & & & & & & 13 \\
          \hline
        \end{tabular}
    \end{adjustbox}
  \end{center}
\end{frame}

\begin{frame}[fragile]{Problema de traveling salesman.}
  \tcajita{Problema}{  Si comenzamos nuestro recorrido en la ciudad $1$ encontrar una ruta que pase por todas las ciudades y que minimice el costo.
  }
\end{frame}

\begin{frame}[fragile]{Variables de decisión.}
  Consideramos las variables de decisión binarias 
  \[
    x_{ij} \in \{0,1\}, \ \ \ \forall i,j. \ \  1 \le i,j \le 8
  \]
  donde: 
  \[
  x_{ij} = 
    \begin{cases}
      1 & \text{si usamos la arista $i$ a $j$ en el camino } \\
      0 & \text{si no} \\
    \end{cases}
  \]
  
  \pause 

  A su vez por ser un problema del traveling salesman agregamos variables auxiliares
  \[
    u_{i} \in \{ 1,2 \dots \}, \ \ \ \forall i. \ \  1 \le i \le 8
  \]
  que nos ayudan a evitar considerar \emph{subtours}.
\end{frame}

\begin{frame}[fragile]{Función objetivo.}
  Queremos minimizar la suma de los costos de cada arista en la ruta.

  \[
    Z = \sum_{i}^{8}\sum_{j=1}^{8} c_{ij}x_{ij}
  \]

  donde $c_{ij}$ es la matriz dada por los costos que aparecen en la tabla anterior.
\end{frame}

\begin{frame}[fragile]{Restricciones.}
  \begin{itemize}
    \item Para cada nodo debe haber una sola arista entrando.
    \pause 
      \[
        \sum_{i}^{8} x_{ij} = 1 \ \ \ j = 1,2, \dots 8
      \]

    \item Para cada nodo debe haber una sola arista saliendo.
    \pause 
      \[
        \sum_{j}^{8} x_{ij} = 1 \ \ \ i = 1,2, \dots 8
      \]
  \end{itemize}
\end{frame}

\begin{frame}[fragile]{Restricciones.}
  \begin{itemize}
    \item Para cada arista tal que su costo sea positivo (exceptuando las aristas incidentes al nodo inicial) agregamos la restricción que evita que hayan subtours en nuestra ruta.
    \pause 
      \[
        u_{i} - u_{j} + 8x_{ij} \le 7, \ \ \ i \neq j, \ \  i,j = 2,3, \dots 8
      \]

    \pause 
    \item Restricciones de positividad para $u_{i}$.
      \[
        u_{i} \ge 0 \ \ \ \ \forall i. \ \ 1 \le i \le 8
      \]
    
    \item Variables de decisión que valen $0$.
      \begin{align*}
        &x_{ii} = 0  & i = 1, \dots, 8 \\
        &x_{14} = 0  & x_{41} = 0      \\
        &x_{15} = 0  & x_{51} = 0      \\
        &\vdots  & \vdots     \\
      \end{align*}
  \end{itemize}
\end{frame}

\begin{frame}[fragile]{Solución.}
  Pasamos a colab para resolverlo.
\end{frame}

\begin{frame}[fragile]{Problema para pensar 1.}
  \textbf{Problema 1}.

  Una empresa de autómoviles tiene dos fábricas, dos galpones y tres concesionarias para vender estos autos.
  Las localizaciones son las siguientes:
  \begin{itemize}
    \item Fábricas: Zarate, Villa María.
    \item Galpones: Rio Cuarto, San Pedro.
    \item Concesionarias: Buenos Aires, Rosario, Córdoba.
  \end{itemize}
  Los autos los hacen en las fábricas y después son llevados a los galpones o bien a las concesionarias. 
  Zarate puedo producir uno 150 autos por semana, y Villa María puede producir 100 por semana.
  Buenos Aires requiere 90 autos por semana; Córdoba requiere 70 autos por semana; y Rosario requiere 60 autos por semana.
  Fabricar un auto cuesta \$25000 y las tablas en la siguiente hoja determinan el precio de trasladar un auto entre dos ciudades. 
  \emph{Determinar como satisfacer los pedidos semanales de la empresa minimizando el costo.}
\end{frame}

\begin{frame}[fragile]{Problema para pensar 1: Tabla.}
  \begin{center}
    \begin{adjustbox}{width = 1.05 \textwidth}
        \begin{tabular}{|c|c|c|c|c|c|}
          \hline
          Origen/Destino & Rio Cuarto & San Pedro & Buenos Aires & Rosario & Córdoba \\
          \hline 
          Zarate      & 1500 & 550  & 750   & 650 & 1400 \\ 
          Villa María & 450  & 1100 & 1250  & 950 & 700 \\
          Río Cuarto  & -    & 950  & 1350  & 1100 & 550 \\
          San Pedro   & 950  &  -   & 950   & 550  & 1200 \\
           \hline
        \end{tabular}
    \end{adjustbox}
  \end{center}
  \alert{Modelar matemáticamente este problema y resolverlo usando Python.}
\end{frame}

\begin{frame}[fragile]{Problema para pensar 2.}
  \textbf{Problema 2.}
  Una compañía proveedora de energía eléctrica necesitará la capacidad de producción que se muestra en la tabla para los próximos 5 años.
  La empresa debe decidir la inversión y operación de distintas fábricas (especificado en las tablas).
  Encontrar una solución al problema de minimizar el costo cubriendo la demanda de energía.
  \begin{columns}
    \begin{column}{0.3 \textwidth}
      \begin{center}
        \begin{adjustbox}{width = 1.10 \textwidth}
            \begin{tabular}{|c|c|}
              \hline
              Año & Demanda energética \\ 
              \hline 
              1 & 80 \\
              2 & 100 \\
              3 & 120 \\
              4 & 140 \\
              5 & 160 \\
              \hline
            \end{tabular}
        \end{adjustbox}
      \end{center}
    \end{column}
    \begin{column}{0.7 \textwidth}
      \begin{center}
        \begin{adjustbox}{width = 1.08 \textwidth}
            \begin{tabular}{|c|c|c|c|}
              \hline
              Fábrica & Capacidad & Costo construcción & Costos corrientes    \\
              \hline
              1 & 70 & 20 & 1,5 \\
              2 & 50 & 16 & 0.8 \\
              3 & 60 & 18 & 1.3 \\ 
              4 & 40 & 14 & 0.6 \\
              \hline
            \end{tabular}
        \end{adjustbox}
      \end{center}
    \end{column}
  \end{columns}
\end{frame}

\begin{frame}[fragile]{Sugerencias.}
  \textbf{Sugerencias.}:
  \begin{itemize}
    \item Pensarlo como un problema de programación entera.
    \item Definir las variables auxiliares para $i = 1, 2,3,4$
      \[
        y_{i} = 
        \begin{cases}
          1 & \text{si construimos la fábrica $i$} \\
          0 & \text{si no} \\
        \end{cases}
      \]
    \item Definir las variables auxiliares para $i=1,\dots,4$ y $j = 1, \dots, 5$.
      \[
        \alpha_{ij} = 
        \begin{cases}
          1 & \text{la fábrica $i$ estuvo en funcionamiento en el año $j$} \\
          0 & \text{si no} \\
        \end{cases}
      \]
  \end{itemize}
  \alert{Modelar matemáticamente este problema y resolverlo usando Python.}
\end{frame}

\end{document}
